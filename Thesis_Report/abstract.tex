\begin{abstract}
\vspace{2cm}
\noindent
Mango leaf diseases are a major threat to crop production, and detecting them early is important to prevent large losses. Manual diagnosis is slow, subjective, and often not available in rural areas. This study explores a deep learning-based system that uses both supervised and semi-supervised learning to classify eight types of mango leaf diseases. Four pretrained models (VGG16, MobileNetV2, DenseNet121, and ResNet101) and a proposed lightweight model were tested on publicly available datasets. In supervised experiments, ResNet101 achieved the highest 99.75\% test accuracy, followed closely by the proposed model with 99.00\%. Semi-supervised learning improved lightweight models like MobileNetV2, increasing accuracy from 91.00\% to 97.50\%, showing the benefit of using unlabeled data when labelled data are limited. Deeper models such as ResNet101 showed lower performance in semi-supervised learning due to errors in pseudo-labels. A mobile application called \textit{PlantDoc Advisor} was built using the proposed model to detect diseases in real time, making it useful for farmers in the field. Overall, the proposed model gave the best balance of accuracy, stability, and efficiency, making it suitable for real agricultural use. These results show that deep learning can help provide fast, reliable, and low-cost plant disease detection to support precision agriculture.

\end{abstract}