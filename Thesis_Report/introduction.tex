\chapter{Introduction}

Mango (\textit{Mangifera indica L.}) is a vital fruit crop in tropical and subtropical regions, particularly in South Asian countries such as Bangladesh and India, where it contributes significantly to agricultural income and rural livelihoods. However, mango cultivation is frequently threatened by various foliar diseases, including Anthracnose, Powdery Mildew, Bacterial Canker, Die Back, and Gall Midge. These diseases can cause substantial yield losses if not detected and managed promptly. Therefore, accurate and early disease detection is critical for sustainable mango production.\\
Recent advancements in deep learning, especially Convolutional Neural Networks (CNNs), have enabled the development of automated plant disease classification systems using image-based data. Supervised learning approaches, in which models are trained on large sets of labeled images, have demonstrated high accuracy in identifying plant diseases. However, these models require extensive labeled datasets, and generating such data often demands expert knowledge, making the process labor-intensive and costly.\\
Semi-supervised learning (SSL) techniques provide a potential solution by leveraging both labeled and unlabeled data. SSL has gained attention for reducing the dependence on labeled data while maintaining competitive performance. Despite its success in general image classification tasks, SSL has been relatively underexplored in the context of agricultural disease classification, particularly for mango leaf disease detection.\\
This thesis presents a comparative study of supervised and semi-supervised learning methods for mango leaf disease classification. Established pre-trained CNN models, including VGG16, ResNet101, MobileNetV2, and DenseNet121, are trained under both supervised and semi-supervised settings to serve as benchmarks. ResNet101, being a very deep architecture, is evaluated to understand how depth and complexity affect performance under semi-supervised learning, where noisy pseudo-labels may impact learning stability. This comparison helps identify the strengths and limitations of each learning approach when applied to real-world agricultural data.\\
In addition to pre-trained models, custom CNN architectures are designed and evaluated specifically for the mango leaf disease dataset. The aim is to assess whether lightweight or task-specific models can achieve comparable or superior performance to complex pre-trained architectures, particularly when labeled data is limited or semi-supervised learning is applied.\\
To demonstrate the practical applicability of the proposed model, a mobile application named \textit{PlantDoc Advisor} was developed. This app utilizes the proposed CNN model to perform real-time disease detection on mango leaves, enabling farmers and agricultural practitioners to quickly identify diseases in the field. By integrating the model into a mobile platform, \textit{PlantDoc Advisor} provides a convenient and scalable tool for early disease diagnosis, helping to reduce yield losses and support timely intervention.\\
Through these experiments and the development of \textit{PlantDoc Advisor}, this study seeks to identify effective approaches that minimize the need for labeled data without compromising classification accuracy, while also offering a practical solution for real-world agricultural applications.
