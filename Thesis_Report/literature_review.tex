\chapter{Literature Review}

Rizvee et al.\cite{rizvee2023leafnet} developed LeafNet, a CNN model tailored for detecting seven mango leaf diseases using region-specific images from Bangladesh. The model achieved high performance with 98.55\% accuracy and outperformed AlexNet and VGG16. Its lightweight structure and robustness make it suitable for early disease detection in real-world agricultural settings.\\
Rao et al.\cite{rao2021deep} implemented a deep learning approach using a modified AlexNet for detecting mango and grape leaf diseases. With a dataset of over 8,000 images, their system achieved 89\% accuracy for mango and 99\% for grape leaves. Integrated into an Android app (JIT CROPFIX), the model supports real-time disease detection on smartphones.\\
Zhang et al.\cite{zhang2018identification} proposed improved GoogLeNet and CIFAR10 models for automatic recognition of maize leaf diseases. By optimizing network parameters and architecture, their system achieved up to 98.9\% accuracy while reducing training iterations and computational cost, proving effective for real-time agricultural diagnostics.\\
Bhujel et al.\cite{bhujel2022lightweight} introduced a lightweight CNN enhanced with various attention modules (e.g., CBAM) for tomato leaf disease classification. Their CBAM-based model achieved 99.69\% accuracy with drastically fewer parameters than ResNet50, enabling efficient deployment on low-resource devices without compromising detection accuracy.\\
Li and Chao et al.\cite{li2021semi} presented a semi-supervised few-shot learning framework that leverages both labeled and high-confidence pseudo-labeled data for plant disease recognition. Their method, tested on PlantVillage, achieved up to 4.6\% performance improvement, showing strong generalisation even with minimal labeled samples.\\
Amorim et al.\cite{amorim2019semi} addressed the challenge of limited labeled data in agricultural image classification by proposing a semi-supervised learning approach. Their method leverages a small set of labeled samples alongside a large pool of unlabeled images to enhance the training of Convolutional Neural Networks (CNNs). By employing label propagation techniques, the model automatically assigns labels to the unlabeled dataset and improves classification accuracy. The approach was validated on UAV-captured images of soybean leaves and herbivorous pests, demonstrating its effectiveness in supporting pest identification and control in real-world agricultural settings, especially when labeled data is scarce.\\
Arivazhagan and Ligi et al.\cite{arivazhagan2018mango} proposed a CNN-based deep learning approach for automating the identification of mango leaf diseases. Their study focused on five common diseases—Anthracnose, Alternaria leaf spots, Leaf Gall, Leaf Webber, and Leaf Burn—using a dataset of 1200 images of both healthy and diseased leaves. The model achieved an impressive classification accuracy of 96.67\%, demonstrating its effectiveness for real-time disease diagnosis in agricultural settings and its potential to assist in early-stage disease control and monitoring across large farms.\\
Rajbongshi et al.\cite{rajbongshi2021recognition} implemented a transfer learning-based classification framework using several advanced CNN models, including DenseNet201, ResNet50, InceptionV3, InceptionResNetV2, and Xception, to identify mango leaf diseases. Their methodology involved image acquisition, segmentation, and feature extraction to classify various disease classes such as Anthracnose, Gall Machi, Powdery Mildew, and Red Rust. Using a dataset of 1500 mango leaf images, the study concluded that DenseNet201 yielded the best performance, achieving an accuracy of 98.00\%, thereby confirming the reliability of deep learning in precise and scalable disease classification.\\
Ahmed et al.\cite{ahmed2023mangoleafbd} addressed a critical gap in the availability of agricultural datasets by introducing MangoLeafBD, the first publicly available, standardized dataset specifically curated for mango leaf disease classification. The dataset comprises 4000 images of approximately 1800 unique leaves, collected from four mango orchards in Bangladesh. Covering seven major disease classes, this dataset provides a diverse and rich foundation for developing and benchmarking machine learning models in agricultural disease detection. The authors emphasized that although the dataset was sourced in Bangladesh, the diseases represented are prevalent in many mango-growing regions, making the dataset broadly applicable for global research.\\
Mia et al.\cite{mia2020mango} proposed a Neural Network Ensemble (NNE) for Mango Leaf Disease Recognition (MLDR) to address the challenges of detecting mango leaf diseases with the naked eye. The system uses machine learning to monitor different leaf symptoms and automatically classify diseases by comparing new leaf images with trained data. The model achieved an average accuracy of 80\% and provides an automated alternative to traditional manual inspection. Its implementation allows timely disease detection without the presence of an agriculturist, helping to properly treat affected leaves, increase mango production, and support the global market demand.
